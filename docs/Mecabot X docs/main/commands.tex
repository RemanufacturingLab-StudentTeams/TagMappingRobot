% !TeX spellcheck = en_GB
\section{Commands}\label{sec:commands}
Bellow are the ros2 commands that are shipped with the standard package. It shows what the commands do and how they work.

To run commands on PC you need to download the needed binary's. Appendix \ref{sec:appendix_copy_bin} tells you how to download the binary from the Mecabot X. Appendix \ref{sec:appendix_compile} show how to compile the binary's yourself.

 If a command needs to be run on the laptop the ''script box'' will be \colorbox{Gainsboro}{grey}. If it needs to be run on the Mecabot the ''script box'' will be \colorbox{NavajoWhite}{yellow}.

\subsection{wheeltec\_rviz2}\label{sec:rviz}

RVIZ2 is an acronym for ROS Visualization 2\cite{RVIZ2}. It is a sophisticated visualization tool developed for ROS 2. It offers a graphical interface that enables users to visualize various aspects of a robot's state, including sensor data, the robot's position, and its planned navigation path.

RVIZ2 should be run on a laptop that is connected to the same \texttt{ROS\_DOMAIN\_ID} and on the same network as the Mecabot.

Run the following command on your laptop:

\begin{tcolorbox}[colback=Gainsboro]
\begin{minted}{bash}
source ~/wheeltec_ros2/install/setup.bash
export ROS_DOMAIN_ID=10
ros2 launch wheeltec_rviz2 wheeltec_rviz.launch.py
\end{minted}
\end{tcolorbox}

This should run a RVIZ window. Figure \ref{fig:RVIZ2} shows an example.

\begin{figure}[H]
	\centering
	\includegraphics[width=0.8\linewidth]{./pictures/RVIZ2.png}
	\caption{Empty RVIZ2 window}
	\label{fig:RVIZ2}
\end{figure}

\subsection{wheeltec\_keyboard}


\subsubsection*{Command}

\begin{tcolorbox}[colback=NavajoWhite]
\begin{minted}{bash}
ros2 run wheeltec_robot_keyboard wheeltec_keyboard
\end{minted}
\end{tcolorbox}

\subsubsection*{Arguments}

None


\subsection{wheeltec\_joy}

\subsubsection*{Command}

\begin{tcolorbox}[colback=NavajoWhite]
\begin{minted}{bash}
ros2 launch wheeltec_joy wheeltec_joy.launch.py
\end{minted}
\end{tcolorbox}

\subsubsection*{Arguments}

None

\subsection{turn\_on\_wheeltec\_robot}
These commands can be used manually. But other commands already do this in the background when booting up.

\subsubsection*{Command}

\begin{tcolorbox}[colback=NavajoWhite]
\begin{minted}{bash}
ros2 launch turn_on_wheeltec_robot <ARGUMENT>
\end{minted}
\end{tcolorbox}

\paragraph{Most used:}
The following arguments are most used. The rest of the commands are shown in the arguments section.

Completely turn on the Mecabot.
\begin{tcolorbox}[colback=NavajoWhite]
\begin{minted}{bash}
ros2 launch turn_on_wheeltec_robot turn_on_wheeltec_robot.launch.py
\end{minted}
\end{tcolorbox}


Turns on the LiDAR:
\begin{tcolorbox}[colback=NavajoWhite]
\begin{minted}{bash}
ros2 launch turn_on_wheeltec_robot wheeltec_lidar.launch.py
\end{minted}
\end{tcolorbox}

\textbf{Turns on sensors:}

\begin{tcolorbox}[colback=NavajoWhite]
\begin{minted}{bash}
ros2 launch turn_on_wheeltec_robot wheeltec_sensors.launch.py
\end{minted}
\end{tcolorbox}

\subsection{wheeltec\_cartographer}

Used to create a map.

\begin{tcolorbox}[colback=NavajoWhite]
	\begin{minted}{bash}
ros2 launch wheeltec_cartographer cartographer.launch.py
	\end{minted}
\end{tcolorbox}

Create map with blind-spots

\begin{tcolorbox}[colback=NavajoWhite]
	\begin{minted}{bash}
ros2 launch wheeltec_cartographer cartographer_with_blindspots.launch.py
	\end{minted}
\end{tcolorbox}


\subsection{nav2\_map\_saver}

Save a map created using cartographer

\begin{tcolorbox}[colback=NavajoWhite]
	\begin{minted}{bash}
ros2 run nav2_map_server map_saver_cli -f <direcotry path>
	\end{minted}
\end{tcolorbox}


\subsection{wheeltec\_nav2}

\begin{tcolorbox}[colback=NavajoWhite]
	\begin{minted}{bash}
ros2 launch wheeltec_nav2 wheeltec_nav2.launch.py
	\end{minted}
\end{tcolorbox}

\texttt{use\_sim\_time:=} can be True or False if you want to simulate the nav2 function.

\texttt{maps:=} Directory where the map is located.

\subsection{lidar\_filter}

Turn on a LiDAR filter.

\begin{tcolorbox}[colback=NavajoWhite]
	\begin{minted}{bash}
ros2 launch lidar_filter lidar_filter.launch.py
	\end{minted}
\end{tcolorbox}

\newpage
