% !TeX spellcheck = en_GB
\subsection{Miscellaneous}

\subsubsection{Add SSH key to Mecabot X}\label{sec:appendix_add_ssh}
By adding your ssh key to the Mecabot X you don't need to put in your password every time you're connecting.

\paragraph{Add config file}
The config file will create a alias so you don't need to type the user and ip every time you are connecting.

On your pc run the following command:

\begin{tcolorbox}[colback=Gainsboro]
	\begin{minted}{bash}
nano ~/.ssh/config
	\end{minted}
\end{tcolorbox}

put the following data in this document:

\begin{tcolorbox}[colback=Gainsboro]
\begin{minted}{yaml}
Host dongguan
  HostName 10.35.4.35
  User wheeltec
  Port 22
\end{minted}
\end{tcolorbox}

\textbf{Note: The amount of spaces is important. You need two.}

The ip should be changed based on the ip of the Mecabot X.

Afterwards you can type the following to connect to the Mecabot X:

\begin{tcolorbox}[colback=Gainsboro]
	\begin{minted}{bash}
ssh dongguan
	\end{minted}
\end{tcolorbox}

\paragraph{Creating SSH key}

If you already have a SSH key you can skip this step

type the following command to create a new SSH key.

\begin{tcolorbox}[colback=Gainsboro]
	\begin{minted}{bash}
ssh-keygen
	\end{minted}
\end{tcolorbox}

It will ask for the name, size and password. You can press enter to leave the settings default.

\paragraph{Send SSH key to the Mecabot X}

To copy your key to the Mecabot X follow the following command:

\begin{tcolorbox}[colback=Gainsboro]
\begin{minted}{bash}
ssh-copy-id wheeltec@10.35.4.35
\end{minted}
\end{tcolorbox}

Now, you won't need to put in the password when connecting.

\subsubsection{Updating .bashrc}\label{sec:appendix_bashrc}

.bashrc is a file with commands that is run every time you open a terminal. By adding the ''source'' and ''export'' command you don't need to set them every time.

To do this run the following command: (This can be done on your pc or the Mecabot X)

\begin{tcolorbox}[colback=Gainsboro]
	\begin{minted}{bash}
echo "source ~/wheeltec_ros2/install/setup.bash" >> ~/.bashrc
echo "export ROS_DOMAIN_ID=10" >> ~/.bashrc
	\end{minted}
\end{tcolorbox}

\textbf{Note: This is already done in the Mecabot X.}

\subsubsection{Copy binary's from Mecabot X}\label{sec:appendix_copy_bin}

If you want to edit the source code locally or you want the source code you'll need the binary's from the Mecabot X.

To get these binary's from the Mecabot X there are two options. It can be done through a graphical interface or through the command line interface.

\paragraph{Graphical} Graphical is the easiest way to copy the data to your pc.

\begin{enumerate}
	\item connect to the same WiFi network as the Mecabot X.
	\item open file explorer on your PC.
	\item type the following in the top bar your file explorer: ''sftp://wheeltec@10.35.4.25''
	\begin{itemize}
		\item Change the ip to the current ip of the Mecabot X.
	\end{itemize}
	\item Type the password in the dialog box.
	\item now you can copy ''wheeltec\_ros2.tgz'' to your pc.
	\item extract this zip file to your home directory (\textit{\~{}/}).
\end{enumerate}

\paragraph{Command line}

You can also copy the needed data using the command line. Follow the these steps to copy the data.

Open a terminal on your pc and type the following command's

\begin{tcolorbox}[colback=Gainsboro]
\begin{minted}{bash}
cd ~
scp wheeltec@10.35.4.35:~/wheeltec_ros2.tgz
tar -zxvf wheeltec_ros2.tgz
\end{minted}
\end{tcolorbox}

\subsubsection{Change ROS model}

ROS needs to know a few thinks from the robot to work properly. The following are the primary properties.

\begin{itemize}
	\item The location and orientation of the LiDAR.
	\item The location and orientation of the 3D camera.
	\item The location and size of the tires.
	\item the size of the main body
\end{itemize}

These properties are located in a URDF file\cite{URDF}. The URDF files are located in the following folder '' wheeltec\_ros2/src/transbot/core/wheeltec\_robot\_urdf/urdf''.

Here you can create a new URDF file. Currently the system uses the ''dongguan.urdf'' file.

\paragraph{Change URDF file}

If you want to make changes to the model I recommend copying the existing URDF file and making changes to the copy.

In the following file you can add the model so the Mecabot X can use it. ''wheeltec\_ros2/src/transbot/core/turn\_on\_wheeltec\_robot/launch/robot\_mode\_description.launch.py''


When changing these files you need to re-compile the software.

\subsubsection{Set LiDAR blind spots}

The system has a LiDAR filter. This can be used so the LiDAR ignores stuff that is mounted on the back. The code is located at: ''wheeltec\_ros2/src/dongguan/lidar\_filter''

The code has two blind spot locations. These can be set in degrees.

\textbf{Note: 0 degrees is when the LiDAR is looking back.}

The LiDAR filter takes the info from the ''scan'' topic and sends the filtered data tot the ''scanned'' topic.

Below is an example of where you can change the current blind spots.

\begin{tcolorbox}[colback=NavajoWhite]
\begin{minted}{c++}
// Blindspot in Degrees. Change to the angle you need.
// When changing these values you need to run "colcon build" to recompile
blind_1_start_ = 317.0 * M_PI / 180.0;  // Start blindspot (in degrees)
blind_1_end_ = 337.0 * M_PI / 180.0;    // End blindstpot (in degrees)

blind_2_start_ = 23.0 * M_PI / 180.0;
blind_2_end_ = 43.0 * M_PI / 180.0;
\end{minted}
\end{tcolorbox}

